
\section{Anthropologie}


\subsection{Fragestellung der philosophischen Anthropologie: Wesen des Menschen}
\begin{itemize}
    \item Zentrale Leitfrage: Was ist der Mensch?
    \item Ziel: das Besondere des Menschen gegenüber Tieren, Maschinen, Göttern etc. herauszuarbeiten
    \item Interdisziplinär: Bezieht Biologie, Psychologie, Soziologie, Theologie u.a. mit ein 
    \item Wichtige Unterfragen:
    \begin{itemize}
        \item Ist der Mensch frei?
        \item Was macht ihn moralisch verantwortlich?
        \item Ist der Mensch ein Vernunft- oder Triebwesen?
    \end{itemize}
\end{itemize}

\subsection{Selbstverständnis des Menschen}
\begin{itemize}
    \item Der Mensch denkt über sich selbst nach - das unterschiedet ihn von anderen Lebewesen
    \item Entwicklung von Identität, Selbstbewusstsein, Werte, Lebensentwurfen
    \item Wandelbar: Selbstverständnis hängt von Epoche, Kultur, Religion und Wissenschaft ab (z.B. früher Geschöpf Gottes - heute evolutionäres Produkt)
\end{itemize}

\subsection{Kultur}
\begin{itemize}
    \item Alles, was der Mensch nicht von Natur aus, sondern durch Gestaltung, Lernen und Weitergabe hervorbringt
    \item Dazu zählen Sprache, Technik, Religion, Kunst, Institutionen usw.
    \item Kultur ist notwendig, um die Mängel der Natur zu kompensieren ($\rightarrow$ Gehlen)
\end{itemize}


\subsection{Arnold Gehlen (1904 - 1976)}
\point{Mängelwesen}
\begin{itemize}
    \item Der Mensch ist im Vergleich zu Tieren ein biologisch unzureichend ausgestaltet (keine Krallen, kein Fell, kein Instiktverhalten, usw.)
    \item Diese Mängel sind aber die Bedingung für seine Freiheit und Entwicklung
    \item Folge: Der Mensch muss seine Umwelt aktiv gestalten, nicht nur anpassen
\end{itemize}

\point{Von Natur aus Kulturwesen}
\begin{itemize}
    \item Um zu überleben, muss der Mensch eine zweite Natur schaffen: die Kultur
    \item Institutionen (Famlilie, Staat, Religion etc.) helfen, den Menschen zu entlasten und zu stabilisieren
    \item Kultur ist notwendig, nicht freiwillig
\end{itemize}

\point{Konzept der Weltoffenheit}
\begin{itemize}
    \item Der Mensch ist nicht festgelegt auf eine bestimmte Umwelt (wie Tiere mit Instinkten)
    \item Er ist weltoffen, also fähig, sich in verschiedene Umwelten anzupassen
    \item Diese Offenheit bedeutet aber auch: Unsicherheit, Entscheidung, Verantwortung
\end{itemize}
