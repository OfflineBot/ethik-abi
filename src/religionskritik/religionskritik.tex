
\section{Religionskritik}

\point{Begriffe:}
\begin{itemize}
    \item Religion / Religiosität
    \item Grundlagen der Religionskritik
    \item Theodizee
\end{itemize}

\point{Religionskritische Positionen} 
\begin{itemize}
    \item Ludwig  Feuerbach:
    \begin{itemize}
        \item Gott als Projektion unserer Vorstellungen
        \item Gott ist das ausgesrpchene Selbst des Menschen
        \item Theologie ist damit Anthropologie
    \end{itemize}
    \item Karl Marx:
    \begin{itemize}
        \item Marx' Kritik an den herrschenden sozio-ökonomischen Verhältnissen
        \item ("Entfremdung")
        \item Religion ist der "Seufzer der bedrängten Kreatur"
        \item Religion ist das "Opium des Volkes"
        \item Materialismus
    \end{itemize}
    \item Sigmund Freud:
    \begin{itemize}
        \item Grundlagen: Freuds Vorstellung über Psyche 
        \item Über-Ich, Ich, Es (Unterbusstsein)
        \item Religion als Illusion
        \item Religion als Neurose (als Reaktion auf die kindliche Hilf- udn Ratlosigkeit des Menschen) 
        \item Gott ist der "Übervater"
    \end{itemize}
\end{itemize}
