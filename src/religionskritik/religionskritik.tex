
\section{Religionskritik}


\subsection{Religion/Religiosität}
\begin{itemize}
    \item \textbf{Religion:} System von Glaubensüberzeugungen, Praktiken und Lebenshaltungen, das sich auf das Göttliche oder Tanszendente bezieht
    \item \textbf{Religiosität:} individuelle Ausprägung religiösen Empfindens oder Denkens, unabhängig von institutioneller Bindung
    \item Religion kann Orientierung, Sinn, Gemeinschaft und moralische Leitlinie bieten
\end{itemize}


\subsection{Grundlagen der Religionskritik}
\begin{itemize}
    \item Religionskritik kann verschiedene Aspekte angreifen:
    \begin{itemize}
        \item \textbf{Inhaltlich:} Kritik an religiösen Aussagen über Gott, Welt, Moral
        \item \textbf{Psychologisch:} Religion als Ausdruck psychischer Bedürfnisse (z.B. Freud)
        \item \textbf{Soziologisch/politisch:} Religion als Instrument der Herrschaft (z.B. Marx)
        \item \textbf{Anthropologisch}: Religion als Projektion menschlicher Eigenschaften (z.B. Feuerbach)
    \end{itemize}
    \item Ziel ist oft die \textbf{Entmythologisierung} oder \textbf{Säkularisierung}
\end{itemize}

\subsection{Theodizee}
\begin{itemize}
    \item Frage anch der \textbf{Gerechtigkeit Gottes} angesichts des Bösen und Leids in der Welt
    \item "Wie kann ein allmächtiger, allgütiger Gott Leid und Böses zulassen?"
    \item Klassisches Problem der Glaubensverteidigung, insbesondere im Christentum
    \item Relevanter Hintergrund für die Religionskritik (z.B. "Leid widerlegt die Vorstellung eines guten Gottes")
\end{itemize}

\subsection{Religionskritische Positionen}

\subsubsection{Ludwig Feuerbach (1804 - 1872)}
\begin{itemize}
    \item Frühsozialistischer Philosoph
    \item \textbf{Gott als Projektion:} Menschen schreiben Gott ihre eigenen idealisierten Eigenschaften zu \\
    $\rightarrow$ "Gott ist as ausgesprochene Selbst des Menschen"
    \item Der Mensch verehrt sein eigenes Wesen, das er entfremdet als Gott vorstellt
    \item \textbf{Theologie = Anthropologie:} Aussagen über Gott sind in Wahrheit Aussagen über Menschen
    \item Ziel: Selbstverwirklichung durch Rücknahme der Projektion
\end{itemize}


\subsubsection{Karl Marx (1818 - 1883)}
\begin{itemize}
    \item Religionskritik eingebettet in seine \textbf{Gesellschafts- ud Kapitalismuskritik}
    \item Religion ensteht aus \textbf{sozialem Leid und Entfremdung}
    \item Berühmtes Zitat: "Religion ist der Seufzer der bedrängten Kreatur, das Gemüt einer herzlosen Welt, wie sie der Geist geistloser Zustände ist. Sie ist das Opium des Volkes."
    \item Religion tröstet, lenkt aber von tatsächlichen gesellschaftlichen Problemen ab
    \item \textbf{Materialismus:} Das Bewusstsein (inkl. Religion) ist Produkt materieller Verhältnisse
    \item Ziel: Religion überwinden durch Veränderung der ökonomischen Verhältnisse
\end{itemize}

\subsubsection{Sigmund Freud (1856 - 1939)}
\begin{itemize}
    \item Begründer der Psychoanalyse
    \item Modell der Psyche: \textbf{Es - Ich - Über-Ich}
    \item Religion als Ausdruck psychischer Mechanismus: \\
    $\rightarrow$ Wunsch nach Schutz, Ordnung, Autorität $\rightarrow$ \textbf{Gott als Übervater}
    \item Relgion = kollektive \textbf{Zwangsneurose:} Reaktion auf kindliche Hilflosigkeit
    \item \textbf{Illusion:} Religion gibt vor, etwas Wahres zu sein, ist aber Wunschprojektion
    \item Ziel: Befreiung durch wissenschaftliche Aufklärung und seelische Reife
\end{itemize}

