
\section{Grundbegriffe}


\subsection{Ethik}

\begin{itemize}
    \item Wissenschaft vom moralischen Handeln
    \item Systematische Relfexion über moralische Urteile und Handlungen
    \item Fragt z.B.:
    \begin{itemize}
        \item Was soll ich tun?
        \item Was ist ein gutes Leben?
        \item Was ist gerecht?
    \end{itemize}
    \item Unterscheidung:
    \begin{itemize}
        \item \textbf{Deskriptive Ethik:} beschreibt moralische Systeme (z.B. in Kulturen)
        \item \textbf{Normative Ethik:} entwickelt moralische Maßstäbe (z.B. Kant, Utilitarismus)
        \item \textbf{Metaethik:} fragt nach Bedeutung moralischer Begriffe (z.B. Was heißt "gut"?)
    \end{itemize}
\end{itemize}

\subsection{Moral}

\begin{itemize}
    \item Gesamtheit der geltenden Wertvorstellungen, Normen und Regeln einer Gesellschaft
    \item Praktisch gelebte Ethik
    \item Bezieht sich auf konkrete Verhaltensweisen, z.B. "Mann soll nicht lügen"
    \item Moral ist oft kulturell geprägt und historisch wandelbar
\end{itemize}

\subsection{Werte und Normen}

\begin{itemize}
    \item \textbf{Werte:} abstrakte Zielvorstellung, was als wünschenswert gilt (z.B. Freiheit, Gerechtigkeit, Solidarität)
    \item \textbf{Normen:} konkrete Handlungsanweisungen, die sich aus Werten ableiten (z.B. "Du sollst nicht stehlen")
    \item Werte begründen Normen, Normen sichern Werte im Alltag
\end{itemize}

\subsection{Gut (verschiedene Bedeutungen)}

\begin{itemize}
    \item \textbf{Instrumentell gut:} etwas ist Mittel zum Zweck (z.B. Messer schneidet gut)
    \item \textbf{Pragmatisch gut:} etwas funktioniert oder ist zweckgemäß (z.B. guter Plan)
    \item \textbf{Moralisch gut:} Handlung entspricht moralischen Maßstäben (z.B. aus Pflicht helfen)
\end{itemize}

$\rightarrow$ Wichtig: In der Ethik geht es nicht umd Nützlichkeit, sondern um moralische Qualität

\subsection{Ethik als Teilgebiet der Philosophie}
\begin{itemize}
    \item Teil der praktmatischen Philosophie (im Gegensatz zur theoretischen Philosophie)
    \item Ziel: Begründung und Relfexion von Normen, Werten, Pflichten, Rechten
    \item Verwandte Disziplinen:
    \begin{itemize}
        \item \textbf{Rechtsphilosophie:} (Was ist gerecht?)
        \item \textbf{Politische Philosophie:} (Wie soll die Gesellschaft organisiert sein?)
        \item \textbf{Anthropologie:} (Was ist der Mensch?)
    \end{itemize}
    \item Ethik ist nicht religiös oder dogmatisch gebunden - sondern rational, argumentativ, kritisch
\end{itemize}


\subsection{Ethik als Teilgebiet der Philosophie}
\begin{itemize}
    \item Richtschnur für Verantwortungsvolles Verhalten
    \item Was ist gut?
    \item Was ist schlecht?
\end{itemize}

