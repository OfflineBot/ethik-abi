
\section{Utilitarismus}

\subsection{Hedonistisches Prinzip}
\begin{itemize}
    \item Das Gute ist das Lustvolle; Ziel ist die Maximierung von Lust bzw. Freude und die Minimierung von Leid
    \item "Lust" kann körperlich, emotional oder geistig verstanden werden (abhängig von Bentham/Mill)
\end{itemize}

\subsection{Konsequenzenprinzip}
\begin{itemize}
    \item Die moralische Richtigkeit einer Handlung wird ausschließlich anhand ihrer Folgen beurteilt
    \item Gute Handlung = Handlung mit besten Folgen
\end{itemize}

\subsection{Utilitätsprinzip}
\begin{itemize}
    \item Nützlichkeit als Maßstab für moralische Handeln
    \item Moralisch richtig ist, was das größtmögliche Glück für die größtmögliche Zahl schafft
\end{itemize}

\subsection{Universalistisches Prinzip}
\begin{itemize}
    \item Jeder wird gleich berücksichtigt, keine Sonderstellung einzelner 
    \item Interessen aller Betroffenen zählen gleich (z.B. auch Tiere bei Singer)
\end{itemize}

\subsection{Hedonistisches Kalkül (Anwendung und Kritik)}
\begin{itemize}
    \item Von \textbf{Bentham} entwickelt: Versucht, Lust/Unlust rechnerisch zu erfassen
    \item Kriterien: Intensität, Dauer, Sicherheit, Nähe, Fruchtbarkeit, Reinheit, Ausmaß
    \item \textbf{Kritik:}
    \begin{itemize}
        \item Quantifizierung von Lust problematisch
        \item Vernachlässigt Gerechtigkeit, Menschenrechte, Würde
        \item Führt ggf. zu Konflikten mit Werten in Entscheidungen (z.B. Minderheit wird geopfert)
        \item Keine klare Gewichtung zwischen verschiedenen Kriterien; Konsequenzen lassen sich nicht immer absehen
    \end{itemize}
\end{itemize}


\subsection{Personen}


\subsubsection{Jeremy Bentham (quantitativer Utilitarismus)}

\begin{itemize}
    \item Fokus auf \textbf{Menge} der Lust, nicht deren Qualität
    \item Alle Freuden gleichwertig, nur quantitativ unterscheidbar
    \item Zitat: "Prejudice apart, the game of push-pin is of equal value with the arts and sciences of music and poetry." \\
        $\rightarrow$ Alles, was Freude bringt, zählt gleich viel
    \item Einführung des \textbf{hedonistisches Kalküls}
\end{itemize}


\subsubsection{John Stuart Mill (qualitativer Utilitarismus)}
\begin{itemize}
    \item Reagiert kritisch Bentham, entwickelt, Theorie weiter
    \item Unterscheidet zwischen \textbf{höheren} (geistigen) und \textbf{niederen} (körperlichen) Freuden
    \item Zitat: "Es ist besser, ein unzufriedener Mensch zu sein als ein zufriedenes Schwein, besser ein unzufriedener Sokrates als ein zufriedener Narr." \\
    $\rightarrow$ Qualität wichtiger als bloße Quantität
    \item Betonung der Bildung und Kultur als Grundlage für "bessere" Lust
\end{itemize}


\subsubsection{Peter Singer (Präferenzutilitarismus)}
\begin{itemize}
    \item Reagiert auf Mill, erweitert Utilitarismus über hedonistisches Lustprinzip hinaus
    \item Moralisch richtig ist, was die \textbf{Präferenzen (Interessen)} aller Betroffenen am besten erfüllt
    \item Grundlage für moderne Tierethik und Bioethik
    \item Einführung von \textbf{Personenbegriff}: moralische Berücksichtigung richtet sich nach Fähigkeit zu leiden, Wünsche zu haben (nicht nach Artzugehörigkeit $\rightarrow$ Kritk am \textbf{Speziesismus})
    \item Verterter einer rationalen, Konsequenzorientierten Ethik unter Einschluss nichtmenschlicher Lebewesen
\end{itemize}

