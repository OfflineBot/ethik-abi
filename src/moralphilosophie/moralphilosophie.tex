
\section{Moralphilosophie}


\subsection{Immanuel Kant (1724 - 1804)}
\point{Grundgedanke:}
\begin{itemize}
    \item Moralisch gut ist nicht das Ergebnis, sondern die Gesinnung, aus der heraus gehandelt wird (die Handlung selbst ist das was unbedingt gut ist)
    \item Ein guter Wille ist das einzig unbedingte Gute
\end{itemize}

\point{Begriffe:}
\begin{itemize}
    \item \textbf{Gesinnung:} Die innere Haltung, aus der heraus man handelt
    \item \textbf{Guter Wille:} Wille, der aus Pflicht handelt - unabhängig vom Erfolg
    \item \textbf{Pflicht/Neigung/pflichtgemäß;}
    \begin{itemize}
        \item \textbf{Pflichtgemäß:} Handlung entspricht der Pflicht (z.B. aus Mitleid helfen); \\
        WICHTIG PFLICH MIT PFLICHTGEMÄß TRENNEN!
        \item \textbf{Guter Wille:} Wille, der aus Pflicht handelt - unabhängig vom Erfolg
        \item \textbf{Aus Neigung:} Handlung erfolgt aus Gefühlen oder Eigeninteresse - moralisch irrelevant
    \end{itemize}
    \item \textbf{Maxime:} subjektiver Handlungsgrunsatz
    \item \textbf{Autonomie:} moralisches Selbstgesetzgeben; der Mensch als vernünftiges Wesen bestimmt selbst das moralische Gesetz
    \item \textbf{Aufklärung:} Kant fodert Menschen auf, mutig den Verstand zu gebrauchen und nichts einfach zu glauben. \\
    $\rightarrow$ von Gott lösen und Verantwortung für eigenes Handeln übernehmen
    \item \textbf{Intelligbible Welt:} Welt des Denkenden, Freiheit, Vernunft - Gegesatz zur sinnlichen Welt
    \item \textbf{Deontologische Ethik:} Pflichtethik, die die Pficht (nicht Folgen) ins Zentrum stellt
    \item \textbf{Goldene Regel:} "Was du nicht willst, was man dir tut, das fug auch keinem andern zu." \\
    $\rightarrow$ Ähnlichkeiten zur Universalisierungsformel, aber nicht identisch mit Kants Begründung
\end{itemize}

\point{Kategorischer Imperativ:}
\begin{itemize}
    \item \textbf{Kategorisch} = unbedingt gültig (im Gegensatz zu "hypothetisch" = "wenn ... dann ...")
    \item \textbf{Imperativ:} Gebotsform
    \item Der \textbf{kategorische Imperativ} ist ein Prüfungsverfahren für moralische maximen (Grundsätze des Handelns)
\end{itemize}

\point{Formeln des kategorischen Imperativs (wichtig für Anwendung):}
\begin{enumerate}
    \item \textbf{Grundformel (Universalisierbarkeit)} \\
    "Handle nur nach derjenigen Maxime, durch die du zugleich wollen kannst, dass sie ein allgemeines Gesetz werde." \\
    $\rightarrow$ Stell dir vor, alle würden so handeln - wäre das widerspruchsfrei möglich?
    \item \textbf{Naturgesetzformel} \\
    "Handle so als ob die Maxime deiner Handlung durch deinen Willen zum allgemeinen Naturgesetz werden sollte." \\
    $\rightarrow$ Pruft die Verallgemeinerbarkeit wie ein Naturgesetz (ohne Ausnahme möglich?)
    \item \textbf{Menschheitszweckformel} \\
    "Handle so, dass du die Menschheit, sowohl in deiner Person als in der Person eines jeden anderen, jederzeit zugleich als Zweck, niemals bloß als Mittel brauchst." \\
    $\rightarrow$ Menschen dürfen nicht instrumentalisiert werden; jeder Mensch hat Würde
\end{enumerate}

\point{Anwendung und Grenzen:}
\begin{itemize}
    \item Klar, rational, inversell anwendbar
    \item Starre Regeln, keine Ausnahmen (z.B. Lügenverbot auch in Extremsituationen)
    \item Berücksichtigt Folgen nicht ausreichend (z.B. bei moralischen Dilemmata)
\end{itemize}

\subsubsection{Moraltestverfahren}
\begin{enumerate}
    \item Dilemma beschreiben
    \item Formuliere zweckrationale der Maxime (immer wenn ... dann ...)
    \item Maxime im Verallgemeinerungstest überprüfen
    \item Auf wiedersprüche untersuchen
    \item Fazit der Anwendung in Praxis
\end{enumerate}


\subsection{Hans Jonas - Das Prinzip Verantwortung}
\point{Grundgedanke:}
\begin{itemize}
    \item Neue Technologien (z.B. Atomkraft, Gentechtnik, KI) erzeugen neue Arten von Risiken $\rightarrow$ Ethik muss auf Zukunft ausgerichtet werden
    \item Erweiterung der traditionellen Ethik (Kant etc.), die primär auf individuelles, gegenwärtiges Handeln bezogen ist
\end{itemize}

\point{Begriffe:}
\begin{itemize}
    \item \textbf{Verantwortung:} Verpflichtung gegenüber der Zukunfst, insbesondere dem Fortbestand menschlichen Lebens
    \item \textbf{Nahethik (Präsenzethik):} Ethik der unmittelbaren Begegnungen (z.B. Kant, klassische Pflichtenethik)
    \item \textbf{Zukunftsethik / Fernverantwortung:} Verantwortung fur nicht-anwesende Personen oder Generationen
    \item \textbf{Sorge-für-Verantwortung:} Wir müssen für die Möglichkeit zukünftiges Lebens sorgen \\
    $\rightarrow$ Leitidee: "Handle so, dass die Wirkungen deiner Handlung verträglich sind mit der Permanenz echten menschlichen Lebens auf der Erde."
\end{itemize}


\point{Kritik an Kant:}
\begin{itemize}
    \item Kants Ethik ist zu individuell, gegenwartsbezogen und berücksichtigt nicht die neuen, globalen Handlungskonsequenzen
    \item Jonas will Kan nicht ablehnen, sonder in ergänzen - um die Verantwortung gegenüber der Zukunft zugewährleisten
\end{itemize}

