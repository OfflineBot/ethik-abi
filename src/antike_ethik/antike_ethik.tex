
\section{Antike Ethik - Aristoteles}

\subsection{Logos}
\begin{itemize}
    \item Vernunft, rationales Denkvermögen des Menschen
    \item Kennzeichenet den Menschen als "vernunftbegabtes Lebewesen" (zoon logon echon)
    \item Grundlage für ethisches Handeln: Nur durch Vernunft kann der Mensch das Gute erkennen und sich tugendhaft verhalten
\end{itemize}

\subsection{Eudaimonia}
\begin{itemize}
    \item Ziel allen menschlichen Handelns: das "gute Leben", das "Glück" im Sinne von Gedeihen oder Gelingen
    \item Kein subjektives Glücksgefühl, sondern objektives Lebensgelingen im Einklang mit Tugend und Vernunft
    \item Wird durch tugendhaftes Handeln in der Gemeinschaft erreicht
\end{itemize}

\subsection{Tugend, dianoethische und ethische Tugenden}
\begin{itemize}
    \item Tugend (aretē): Exzellenz, sittliche Vortefflichkeit
    \item Zwei Arten:
    \begin{itemize}
        \item \textbf{Ethische Tugenden:} Charaktertugenden (z.B. Tapferkeit, Besonnenheit, Großzügigkeit), enstehen durch Gewöhnung
        \item \textbf{Dianoethische Tugenden:} Verstandestugenden (z.B. Weisheit, Klugheit), entstehen durch Belehrung
    \end{itemize}
    \item Ziel ist ein ausgewogenes Handeln durch die richtige Haltung
\end{itemize}

\subsection{Richtige Mitte (mesotes)}
\begin{itemize}
    \item Tugend als Mitte zwischen zwei Extremen (z.B. Tapferkeit = Mitte zwischen Tollkühnheit und Feigheit)
    \item Nicht mathematisch exakt, sondern abhängig von der Situation
    \item Maßstab: vernünftiges Urteil eines tugendhaften Menschen
\end{itemize}

\subsection{Phronesis (praktische Klugheit)}
\begin{itemize}
    \item Fähigkeit, im konkreten Fall das richtige Maß zu erkennen und richtig zu handeln
    \item Wichtige dianoethische Tugend für ethisches Handeln
    \item Verbindet Wissen (Theorie) und Handeln (Praxis)
\end{itemize}

\subsection{Praxis}
\begin{itemize}
    \item Handlen im ethischen Sinne, das uaf ein gutes und tugendhaftes Leben abzielt
    \item Ziel ist nicht bloße Wirkung, sondern das Handeln selbst (Selbstzweck)
    \item Gegensatz zur \textbf{Poiesis} Herstellung
\end{itemize}

\subsection{Theoria}
\begin{itemize}
    \item Kontemplatives Leben, höchste Form menschlicher Tätigkeit
    \item Betrachtung des Wahren, verbunden mit Weisheit (sophia)
    \item Gilt bei Aristoteles als höchste Form der Eudaimonia
\end{itemize}

\subsection{Zoon logon echon / zoon politikon}
\begin{itemize}
    \item \textbf{Zoon logon echon:} Der Mensch ist ein Wesen mit Vernunft
    \item \textbf{Zoon politikon:} Der Mensch ist ein Gemeinschaftswesen (sozial-politisches Wesen)
    \item Nur in der Polis kann der Mensch seine Tugenden entfalten und Eudaimonia erreichen
\end{itemize}

\subsection{Vorstellung von der Seele}
\begin{itemize}
    \item Dreiteilige Seele:
    \begin{itemize}
        \item \textbf{Vegetativ (pflanzlich):} Wachstum, Ernährung - allen Lebewesen gemeinsam
        \item \textbf{Animalisch:} Wahrnehmung, Begehren - mit Tieren gemeinsam
        \item \textbf{Vernünftig (rational):} Denken, Urteilen - spezifisch menschlich 
    \end{itemize}
    \item Ethik bezieht sich auf den vernuftbegabten Teil der Seele
\end{itemize}


