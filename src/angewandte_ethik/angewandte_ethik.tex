
\section{Angewandte Ethik}

\subsection{Anwendung von bekannten moralphilosophischen Theorien und eigenen Überlegungen auf echte (Alltags-)Probleme und Dilemmata}
\begin{itemize}
    \item Ethik soll nicht nur theoretisch bleiben, sonddern praktische Orientierung bieten
    \item Bekannte Theorien (z.B. Utilitarismus, Pflichtethik, Tugendethik) dienen als Werkzeuge zur Analyse konkreter Fälle (z.B. Sterbehilfe, Tierrechte, Klimaschutz)
    \item Eigene moralische Urteile sollten durch Argumentation und Prinzipien gestützt sein, nicht nur durch Intuition oder Gefühle
\end{itemize}

\subsection{Verantwortlich entscheiden}
\begin{itemize}
    \item Verantwortung bedeutet: eigenes Handeln und dessen Folgen reflektieren und vertreten können
    \item Verantwortung sowohl gegenüber einzelnen Betroffenen als auch gegenüber der Gesellschaft, der Zukunft oder der Umwelt
    \item Vorraussetzung: informierte Entscheidung, Berücksichtigung aller relevanten Perspektiven; Außerdem freien Willen und grundsätzlich alle Formen von Freiheit (Handlungsfreiheit..)
\end{itemize}

\subsection{Dilemma}
\begin{itemize}
    \item Entscheidungssituation, in der zwei (oder mehr) moralische Prinzipien miteinander in Konflikt stehen
    \item Jede mögliche Handlung führt zu einem moralisch problematischen Ergebnis 
    \item Beispiel: Soll man lügen, um ein Leben zu retten?
\end{itemize}

\subsection{Abwägung}
\begin{itemize}
    \item Methode zur Lösung von Dilemmata: gegensätzliche moralische Werte oder Pflichten werden gewichtet
    \item Ziel: begründete Entscheidung, welche Pflicht/Voraussetzung im konkreten Fall überwiegt
    \item Beispiel: Abwägung von Autonomie vs. Fürsorgepflicht
\end{itemize}

\subsection{Ambivalenz}
\begin{itemize}
    \item Zwiespältigkeit moralischer Fragen oder Gefühle
    \item In vielen ethischen Problemen gibt es kein klares "richtig" oder "falsch"
    \item Menschen erleben Unsicherheit oder Widerspruch in der moralischen Beurteilung - das ist normal und Teil moralischer Reife
\end{itemize}

\subsection{Relativismusvorwurf}
\begin{itemize}
    \item Kritik: Wenn jede moralische Meinung gleich gültig ist (ethischer Relativismus), dann kann man kein Verhalten mehr als falsch kritisieren (z.B. Menschenrechtsverletzung)
    \item Gefahr: Beliebigkeit und Verlust von Verbindlichkeit in moralischen Fragen
    \item Antwort: Zwischen deskriptivem (Kulturen sind verschieden) und normativem Relativismus (alles ist erlaubt) unterscheiden - in der Philosophie meist Ablehnung des normativen Relativismus
\end{itemize}
